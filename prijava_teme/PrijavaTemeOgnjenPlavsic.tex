\documentclass[a4paper]{article}
\usepackage[OT1, OT2]{fontenc}
\setlength{\textheight}{25cm}
\setlength{\textwidth}{18cm}
\setlength{\topmargin}{-25mm}
\setlength{\hoffset}{-25mm}
\def\zn{,\kern-0.09em,}

\newcommand{\Lat}{\fontencoding{OT1}\fontfamily{cmr}\selectfont}

\begin{document}
\thispagestyle{empty}

\fontfamily{wncyr}
\fontencoding{OT2}\selectfont

\begin{flushleft}
Matematichki fakultet\\
Univerziteta u Beogradu
\end{flushleft}

\bigskip

\begin{center}
\textbf{MOLBA\\
ZA ODOBRAVANJE TEME MASTER RADA
}\end{center}

\bigskip

\begin{flushleft}
Molim da se odobri izrada master rada pod naslovom:
\end{flushleft}

\begin{minipage}{16.5cm}
%%%%%%%%%%%%%%%%%%%%%%%%%%%%%%%%%%%%%%%%%%%%%%%%%%%%%%%%%%%%%%%%%%%%%%%%%%%%%%%
% U donji red upisati naziv master rada umesto teksta: "Naziv master rada"    %
%%%%%%%%%%%%%%%%%%%%%%%%%%%%%%%%%%%%%%%%%%%%%%%%%%%%%%%%%%%%%%%%%%%%%%%%%%%%%%%
\textbf{\textit{\zn Implementacija alata za statichku analizu {\Lat C++14} koda sa predlozima izmena koda \newline korish{}{c1}enjem biblioteka {\Lat LLVM} infrastrukture''}}
\end{minipage}\\
\rule[4mm]{17.5cm}{.05mm}
\begin{flushleft}
\framebox{
\begin{minipage}[t][10cm]{17cm}
%%%%%%%%%%%%%%%%%%%%%%%%%%%%%%%%%%%%%%%%%%%%%%%%%%%%%%%%%%%%%%%%%%%%%%%%%%%%%%%
%  -- unutrasnjost pravougaonika --                                           %
%  Umesto donjeg teksta upisati znacaj i specificni cilj master rada          %
%%%%%%%%%%%%%%%%%%%%%%%%%%%%%%%%%%%%%%%%%%%%%%%%%%%%%%%%%%%%%%%%%%%%%%%%%%%%%%%
\textbf{Znachaj teme i oblasti:}

% 	Umesto donjeg teksta opisati značaj teme i oblasti	%

Standard {\Lat \textit{AUTOSAR C++14}} jedan je od vodec1ih standarda za pravilno pisanje {\Lat \textit{C++}} koda prilikom razvijanja aplikacija za uredjaje sa ugradjenim rachunarom (eng. \textit{{\Lat embedded systems}}) i druge sigurnosno kritichne sisteme [1]. Ovaj standard je posebno vazhan u autoindustriji odnosno za razvoj softvera za automobile.\\
{\Lat LLVM} infrastruktura predstavlja kolekciju modularnih i ponovo iskoristivih kompajlerskih tehnologija i alata [2].
{\Lat LLVM} infrastruktura pruzha podrshku za jednostavno razvijanje kvalitetnih alata za statichku analizu. Ovo je omoguc1eno posebno dizajniranim bibliotekama za razvoj alata ali i prednjim delom (\textit{eng. {\Lat frontend}}) kompajlera za {\Lat \textit{C}} familiju jezika, {\Lat \textit{Clang}}-om, kog karakterishe ekspresivna dijagnostika koda.

\textbf{Specifichni cilj rada:}

% 	Umesto donjeg teksta opisati specifični cilj master rada %

Korish{}{c1}enjem biblioteka kompajlerske infrastruktere {\Lat LLVM} bic1e razvijen alat za statichku analizu {\Lat \textit{C++14}} koda. Alat c1e proveravati
saglasnost sa podskupom pravila iz standarda {\Lat \textit{AUTOSAR C++14}} koja se odnose na deklaracije i deklaratore. Uz prijavljivanje upozorenja
za konstrukte koji nisu u skladu sa ovim standardom alat c1e da predlazhe adekvatne izmene koda. Alat c1e podrzhavati opcije komandne linije
kojim se mozhe izabrati podskup pravila sa kojim c1e se izvrshiti provera izvornog koda kao i opciju za primenu predlozhenih izmena na izvorni k\^{o}d.

\textbf{Literatura:}

{\Lat [1] AUTOSAR. Guidelines for the use of the C++14 language in critical and
safety-related systems, 2017.

[2] LLVM, The LLVM Compiler Infrastructure, on-line at: https://llvm.org/}
% 	Umesto donjeg teksta навести друге битне информације %


% \\
% \\
% \\


% \begin{tabular}{|c|c|}
%    \hline
%    \multicolumn{2}{|c|}{\textbf{Uput{}stvo za pisanje nashih slova}} \\
%    \hline\hline
% 	ligatura & rezultujuc1i simbol  \\
%    \hline
%    \texttt{{\Lat dj}} & dj \\
%    \hline
%    \texttt{{\Lat Dj}} & Dj \\
%    \hline
%    \texttt{{\Lat zh}} & zh \\
%    \hline
%    \texttt{{\Lat Zh}} & Zh \\
%    \hline
%    \texttt{{\Lat lj}} & lj \\
%    \hline
%    \texttt{{\Lat Lj}} & Lj \\
%    \hline
%    \texttt{{\Lat nj}} & nj \\
%    \hline
%    \texttt{{\Lat Nj}} & Nj \\
%    \hline
%    \texttt{{\Lat c1}} & c1 \\
%    \hline
%    \texttt{{\Lat C1}} & C1 \\
%    \hline
%    \texttt{{\Lat ch}} & ch \\
%    \hline
%    \texttt{{\Lat Ch}} & Ch \\
%    \hline
%    \texttt{{\Lat d2}} & d2 \\
%    \hline
%    \texttt{{\Lat D2}} & D2 \\
%    \hline
%    \texttt{{\Lat sh}} & sh \\
%    \hline
%    \texttt{{\Lat Sh}} & Sh \\
%    \hline
%    \texttt{{\Lat ts}} & ts \\
%    \hline
%    \texttt{{\Lat t\{\}s}} & t{}s \\
%    \hline
% \end{tabular}


\end{minipage}
}
\end{flushleft}
\vspace{1cm}
%%%%%%%%%%%%%%%%%%%%%%%%%%%%%%%%%%%%%%%%%%%%%%%%%%%%%%%%%%%%%%%%%%%%%%%%%%%%%%%
% u donji red uneti:         ime i prezime, broj indeksa i modul studenta     %
%%%%%%%%%%%%%%%%%%%%%%%%%%%%%%%%%%%%%%%%%%%%%%%%%%%%%%%%%%%%%%%%%%%%%%%%%%%%%%%
\makebox[10cm][c]{\textbf{$$Ognjen Plavshic1$$, $$1014/2019$$, $$R$$}}
%%%%%%%%%%%%%%%%%%%%%%%%%%%%%%%%%%%%%%%%%%%%%%%%%%%%%%%%%%%%%%%%%%%%%%%%%%%%%%%
% u donji red uneti:               ime i prezime mentora                      %
%%%%%%%%%%%%%%%%%%%%%%%%%%%%%%%%%%%%%%%%%%%%%%%%%%%%%%%%%%%%%%%%%%%%%%%%%%%%%%%
Saglasan mentor \makebox[7cm][c]{\textbf{$$prof. dr Milena Vujoshevic1 Janichic1$$}} \\
\rule[4mm]{10cm}{.05mm} \hfill \raisebox{4mm}{\makebox[5cm][l]{.\dotfill.}} \\
\raisebox{1cm}%
[9mm][0mm]{\makebox[10cm][c]{\textit{(ime i prezime studenta, br. indeksa, smer i modul)}}} \\
\makebox[10cm]{ }\\
\vspace{-1cm}\\
\rule[2cm]{6.5cm}{.05mm} \hfill \rule[2cm]{6.5cm}{.05mm}\\
\vspace{-2.4cm}\\
\raisebox{2cm}{\makebox[6.5cm][c]{\textit{(svojeruchni potpis studenta)}}}
\hfill \raisebox{2cm}{\makebox[6.5cm][c]{\textit{(svojeruchni potpis mentora)}}}\\
\vspace{-2cm}\\
%%%%%%%%%%%%%%%%%%%%%%%%%%%%%%%%%%%%%%%%%%%%%%%%%%%%%%%%%%%%%%%%%%%%%%%%%%%%%%%
% u donji red uneti datum podnosenja molbe                                    %
%%%%%%%%%%%%%%%%%%%%%%%%%%%%%%%%%%%%%%%%%%%%%%%%%%%%%%%%%%%%%%%%%%%%%%%%%%%%%%%
\makebox[5.5cm][c]{$<$datum$>$}\makebox[5.5cm]{} Chlanovi komisije\\
%%%%%%%%%%%%%%%%%%%%%%%%%%%%%%%%%%%%%%%%%%%%%%%%%%%%%%%%%%%%%%%%%%%%%%%%%%%%%%%
% POPUNJAVA MENTOR (rucno ili na sledeci nacin):                              %
% u donji red umesto .dotfill. upisati podatke o 1. clanu komisije            %
%%%%%%%%%%%%%%%%%%%%%%%%%%%%%%%%%%%%%%%%%%%%%%%%%%%%%%%%%%%%%%%%%%%%%%%%%%%%%%%
\rule[4mm]{5.5cm}{.05mm}\makebox[5.5cm]{ } 1. \makebox[6cm][l]{.\dotfill.}\\
\vspace{-8mm}\\
\raisebox{4mm}%
[7mm][0mm]{\makebox[5.5cm][c]{\textit{(datum podnoshenja molbe)}}}\makebox[5.5cm]{ }
%%%%%%%%%%%%%%%%%%%%%%%%%%%%%%%%%%%%%%%%%%%%%%%%%%%%%%%%%%%%%%%%%%%%%%%%%%%%%%%
% POPUNJAVA MENTOR (rucno ili na sledeci nacin):                              %
% u donji red umesto .\dotfill. upisati podatke o 2. clanu komisije           %
%%%%%%%%%%%%%%%%%%%%%%%%%%%%%%%%%%%%%%%%%%%%%%%%%%%%%%%%%%%%%%%%%%%%%%%%%%%%%%%
2. \makebox[6cm][l]{.\dotfill.}\\

\vspace{1cm}


\begin{flushleft}
%%%%%%%%%%%%%%%%%%%%%%%%%%%%%%%%%%%%%%%%%%%%%%%%%%%%%%%%%%%%%%%%%%%%%%%%%%%%%%%
% u donji red upisati                 katedru                                 %
%%%%%%%%%%%%%%%%%%%%%%%%%%%%%%%%%%%%%%%%%%%%%%%%%%%%%%%%%%%%%%%%%%%%%%%%%%%%%%%
Katedra \makebox[9.5cm][l]{\textbf{$<$katedra$>$}} je saglasna sa predlozhenom temom.
\vspace{-3mm}
\hspace*{13mm} \rule[2.3cm]{9.5cm}{.05mm}\\
\vspace{-1cm}
%%%%%%%%%%%%%%%%%%%%%%%%%%%%%%%%%%%%%%%%%%%%%%%%%%%%%%%%%%%%%%%%%%%%%%%%%%%%%%%
% POPUNJAVA SEF KATEDRE                                                       %
%%%%%%%%%%%%%%%%%%%%%%%%%%%%%%%%%%%%%%%%%%%%%%%%%%%%%%%%%%%%%%%%%%%%%%%%%%%%%%%
\makebox[6.5cm][c]{} \hfill \makebox[6.5cm][c]{}\\
\rule[4mm]{6.5cm}{.05mm} \hfill \rule[4mm]{6.5cm}{.05mm}\\
\vspace{-5mm}
\makebox[6.5cm][c]{\textit{(shef katedre)}} \hfill \makebox[6.5cm][c]{\textit{(datum odobravanja molbe)}}
\end{flushleft}
\end{document} 
